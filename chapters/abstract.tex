Negli ultimi anni la crescita delle reti ha portato ad un aumento del flusso di dati rendendo così necessario uno sforzo maggiore per catalogare, indicizzare e pulire i dati. 
Quando si progetta una base di dati si tiene conto di alcuni importanti parametri che ne attestano la qualità, come le dipendenze funzionali che sono utilizzate per ridurre le anomalie e migliorare la qualità dei dati.
Le dipendenze funzionali sono utilizzate anche per altri scopi che si sono resi evidenti nell’ultima decade. Infatti,  all’aumento della quantità di dati verificatosi negli ultimi anni è conseguito un aumento inversamente proporzionale della qualità. E' stato pertanto necessario adattare le dipendenze funzionali per individuare le inconsistenze in modo più ampio.
Le dipendenze funzionali rilassate o approssimate (RFD) sono una generalizzazione delle dipendenze funzionali canoniche che le rendono facilmente adattabili ai diversi contesti applicativi. Infatti una RFD può applicarsi solo ad una porzione di un database e, cosa più importante, il concetto di uguaglianza tra valori di tuple è sostituito nelle RFD  con il concetto di similarità. 
Il lavoro di tesi è consistito nell’utilizzare le RFD in un sistema di rilassamento di query (\emph{Query Relaxation}) che consente di ottenere un Result Set più ampio nel caso in cui il risultato di una query non sia soddisfacente. Solitamente ogni DataSet (o Database) contiene un numero abbastanza elevato di RFD, di conseguenza è stato necessario studiare e sviluppare un criterio di selezione e preferenza per le RFD, in modo tale da restituire solo le dipendenze che risultato maggiormente utili nel rilassare una determinata query. Dato che non esiste una dimostrazione che indichi quale criterio produce un risultato migliore, è stato necessario affrontare il problema in modo sperimentale effettuando dei test per individuare il metodo più efficace.
Il concetto ivi sviluppato non riguarda il semplice ampliamento dei range sui vincoli di una query ma bensì, grazie all'utilizzo delle RFD, la possibilità di modificare/sostituire i vincoli della query ottenendo un Result Set più ampio che sia strettamente correlato alla query iniziale.