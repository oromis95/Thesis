Negli ultimi anni la crescita delle reti ha portato ad un aumento del flusso di dati, si è reso così necessario uno sforzo maggiore per catalogare, indicizzare e \textit{pulire} i  dati. 
La branca dell’informatica più inerente a questo problema è quella legata alle Basi di Dati. 
Quando si progetta un Database si tiene conto di alcuni importanti parametri che ne attestano la qualità, in particolare si cercano le \textbf{dipendenze funzionali}. Tali dipendenze vengono infatti utilizzate per per ridurre le anomalie e migliorare la qualità del Database.
Le \textbf{dipendenze funzionali} hanno però anche altri utilizzi, che si sono resi evidenti nell’ultima decade. Dato che all’aumento della quantità di dati verificatosi negli ultimi anni è conseguito un aumento inversamente proporzionale della qualità,è stato necessario  adattare le dipendenze funzionali per individuare le inconsistenze in modo più ampio.
Le \textbf{Dipendenze funzionali rilassate} o approssimate (\textbf{RFD}) sono una generalizzazione delle dipendenze funzionali canoniche, possiedono delle caratteristiche che le rendono più adattabili, infatti la dipendenza può applicarsi solo ad alcune tuple e non a tutto il Database e cosa più importante il concetto di uguaglianza delle FD viene sostituito nelle \textbf{RFD}  con il concetto di similarità. 
Il lavoro di tesi è consistito nell’utilizzare le \textbf{RFD} in un sistema di \textbf{Query Relaxation} che permette di ottenere un Result Set più ampio nel caso in cui il risultato di una query non sia soddisfacente.
Il concetto ivi sviluppato non riguarda il semplice ampliamento dei range sui vincoli di una query, bensì grazie alle \textbf{RFD}  è possibile modificare/sostituire i vincoli della query ottenendo un Result Set più ampio che sia però in qualche modo correlato al Result Set iniziale.