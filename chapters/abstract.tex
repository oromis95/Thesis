Negli ultimi anni la crescita delle reti ha portato ad un aumento del flusso di dati, si è reso così necessario uno sforzo maggiore per catalogare, indicizzare e “pulire” i  dati. 
La branca dell’informatica più inerente a questo problema è quella legata alle Basi di Dati. 
Quando si progetta un Database si tiene conto di alcuni importanti parametri che ne attestano la qualità, in particolare si cercano le dipendenze funzionali. Queste vengono infatti utilizzate per definire i vincoli di integrità in modo da ridurre le anomali e migliorare la qualità del Database.
Le dipendenze funzionali hanno però anche altri utilizzi, che si sono resi evidenti nell’ultima decade dato che all’aumento dei dati è conseguito un aumento inversamente proporzionale della qualità, in questo caso per individuare le inconsistenze in modo più ampio è stato necessario adattare le dipendenze funzionali.
Le dipendenze funzionali rilassate o approssimate (RFD) sono una generalizzazione delle dipendenze funzionali canoniche, possiedono delle caratteristiche che le rendono più adattabili, infatti la dipendenza può applicarsi solo ad alcune tuple e non a tutto il Database e cosa più importante il concetto di uguaglianza delle FD viene sostituito nelle RFD con il concetto di similarità. 
Il lavoro di tesi è consistito nell’utilizzare le RFD in un sistema di Query Relaxation che permetta di ottenere un Result Set più ampio nel caso in cui il risultato di una query non sia soddisfacente.
Il concetto ivi sviluppato non riguarda il semplice ampliamento dei range sui vincoli di una query, bensì grazie alle RFD è possibile modificare/sostituire i vincoli della query ottenendo un Result Set più ampio che sia però in qualche modo correlato al Result Set iniziale.