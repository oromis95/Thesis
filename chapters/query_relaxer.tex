Oggigiorno dato l'ampio utilizzo del World Wide Web, soprattutto riguardo la ricerca e l'estrazione di informazioni, i sistemi di interrogazione hanno assunto ruolo fondamentale.
Normalmente una ricerca può avere due risultati: esito positivo se esiste un oggetto che corrisponde ai parametri cercati ed esito negativo se tale oggetto non esiste.
Dato che nell'ambito Web i database sono molto ampi spesso si possono verificare due problemi principali:
\begin{enumerate}
    \item Risposta vuota: Il Result Set restituito da un interrogazione è vuoto o contiene pochi elementi. In questo caso l'utente potrebbe desiderare di rilassare in qualche modo la query, ottenendo cosi ulteriori risultati che siano in qualche modo correlati con le preferenze espresse inizialmente.
    \item Risposta troppo ampia: Quando la query non è troppo selettiva si rischia di ottenere un Result Set troppo ampio. In questo caso è naturale voler avere qualche sistema che ordini i risultati ottenuti in base ad un criterio di Ranking che dovrebbe dare precedenza ad elementi che siano di  interesse maggiore per l'utente.
\end{enumerate}

Per quanto riguarda il primo problema, la ricerca ha proposto varie soluzioni. L'idea iniziale è quella di ridurre i vincoli della query originale. Alcune soluzioni più raffinate applicano un criterio di ordinamento degli attributi in base alle preferenze dell'utente. Effettuando un query del tipo
\\~\\
\centerline{SELECT * FROM CarDB WHERE Model=Camry and Price$<$10000}
\\~\\
L'utente riceverà in output un Result Set contenente una serie di auto modello "Camry" con prezzo sotto i 10000. Nel caso in cui l'utente non sia soddisfatto del risultato, i motori di ricerca applicano un rilassamento ampliando i vincoli indifferentemente. \\
Qui sorge un problema, nella realtà ampliare in vincolo piuttosto che un altro ha un suo peso. Ad esempio un utente potrebbe essere disposto maggiormente a cambiare modello d'auto piuttosto che aumentare il budget della spesa. Ciò dimostra che è necessario un sistema di rilassamento che dia possibilità all'utente di stabilire quali siano gli attributi più importanti, che quindi non devono essere modificati, e quali siano gli attributi di importanza secondaria, sui quali l'utente è disposto ad adattarsi. 
\section{Tecniche di Relaxing}
\paragraph{Rilassamento banale}
Qualsiasi motore di ricerca implementa delle funzionalità di Relaxing.
Il sistema più semplice è quello di ampliare gradualmente tutti i vincoli di una query, in alcuni casi si arriva anche ad eliminarne alcuni.
Quasi sicuramente questo tipo  di rilassamento produce ulteriori risultati, spesso ne arriva a produrre anche troppi.
\paragraph{Rilassamento basato sulla teoria fuzzy degli insiemi}
Vi alcuni algoritmi \cite{generalfuzzy} \cite{emptyvs} che rilassano i vincoli della query utilizzando un misto di funzioni membership , conoscenza del dominio applicativo e operazione di teoria fuzzy come le $\alpha-cut$
\paragraph{Rilassamento basato su sistemi collaborativi}
Probabilmente questo è il sistema più interessante, se non anche il più utilizzato. Il criterio di rilassamento è basato su sofisticate funzioni di distanza \footnote{Spesso queste funzioni richiedono della conoscenza esterna, ad esempio una funzione per il calcolo delle distanze fra due città necessità di un database contenente tutti questi dati.}, conoscenza della distribuzione interna di un Database. 
Questo tipo di rilassamento è fortemente vincolato dal database a cui si applica. L'ottenimento di risultati migliori è direttamente proporzionale alla quantità di $conoscenza$ della distribuzione del database, questo tipo di dati è estraibile con un forte dispendio computazionale in quanti si richiede l'analisi dell'interno Database.  In questa tipologia di rilassamento rientra il sistema di Query Relaxation sviluppato durante il lavoro di tesi.

\section{Tecniche di cleaning del risultato}
Quando viene restituito un Result Set troppo ampio esistono due approcci principali:
\begin{itemize}
\item Trasformare il Result Set in un albero navigabile, in modo da poter effettuare una potatura.
\item Applicare un ordinamento sul Result Set ottenuto, le tuple che vengono prima devono essere quelle correlate maggiormente con la query iniziale.
\end{itemize}

La funzione di ordinamento può essere effettuata tramite diversi approcci:
\paragraph{Basata su Preferenze}
Alcuni sistemi permettono all'utente di specificare quale parametro possiede importanza maggiore. Si può ad esempio specificare che si sta effettuato una ricerca dove il parametro prezzo deve essere inflessibile. In questo caso ,una volta generato il Result Set , viene effettuato un ordinamento dando precedenza ai parametri più importanti.\cite{query_answering}
\paragraph{Basata su conoscenza indiretta}
Oltre le preferenze degli utenti si tiene conto di alcuni parametri che sono estraibili con delle analisi del Database e delle query su di esso effettuate.
Rientrano in questo campo gli approcci basati su Dipendenze Funzionali Approssimate.
Una tecnica interessante è quella di controllare quale risultato ha maggiore probabilità. Effettuando una ricerca di un prodotto su degli Store online si notano spesso ai primi posti prodotti ,che seppur non corrispondenti completamente alla query, hanno un alto tasso di vendite. 


