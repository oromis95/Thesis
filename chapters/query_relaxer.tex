Oggigiorno dato l'ampio utilizzo del World Wide Web, sopratutto riguardo la ricerca e l'estrazione di informazioni, i sistemi di interrogazione hanno assunto ruolo fondamentale.
Normalmente una ricerca può avere due risultati: esito positivo se esiste un oggetto che corrisponde ai parametri cercati e esito negativo se tale oggetto non esiste.
Dato che nell'ambito web i database sono molto ampi spesso si possono verificare due problemi principali:
\begin{enumerate}
    \item Risposta vuota: Il Result Set restituito da un interrogazione è vuoto o contiene pochi elementi. In questo caso l'utente potrebbe desiderare di rilassare in qualche modo la query, ottenendo cosi ulteriori risultati che siano in qualche modo correlati con le preferenze espresse inizialmente.
    \item Risposta troppo ampia: Quando la query non è troppo selettiva si rischia di ottenere un Result Set troppo ampio. In questo caso è naturale voler avere qualche sistema che ordini i risultati ottenuti in base ad un criterio di Ranking basato sulla corrispondenza con la query iniziale.
\end{enumerate}

Nel primo esempio, la ricerca ha proposto varie soluzioni. L'idea iniziale è quella di ridurre i vincoli della query originale. Alcune soluzioni più raffinate applicato un criterio di ordinamento degli attributi in base alle preferenze dell'utente. Considerando un query del tipo "-CarDB (Model = Camry and Price < 10000) " 