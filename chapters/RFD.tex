\section{Dipendenze funzionali canoniche}
Data una relazione $r$ su uno schema $R(X)$ e due sottoinsiemi di attributi non vuoti $Y$ e $Z$ di $X$,diremo che esiste su $r$ una dipendenza funzionale\footnote{Il termine Dipendenza Funzionale nella tesi viene spesso abbreviato in FD (Functional Dependency).} tra gli attributi $Y$ e $Z$, se, per ogni coppia di tuple $t_1$ e $t_2$ di $r$ aventi gli stessi valori sugli attributi $U$, risulta che $t_1$ e $t_2$ hanno gli stessi valori anche sugli attributi $Z$.
Generalmente indichiamo una dipendenza funzionale tra gli attributi $Y$ e $Z$ con la notazione $Y->Z$ , ogni FD  viene associata ad uno schema.

Dato che le dipendenze funzionali dovrebbero servire a descrivere proprietà significative della applicazione, è necessario separare dalle dipendenze funzionali le dipendenze funzionali banali.
In generale una dipendenza funzionale $Y->A$ è non banale se $A$ non compare tra gli attributi di $Y$.
Un'ultima osservazione va fatta sul legame presente tra il concetto di dipendenza funzionale e il concetto di vincolo di chiave. Possiamo dire che una dipendenza funzionale $Y->Z$ su uno schema $R(X)$ degenera nel vincolo di chiave se l’unione di $Y$ e $Z$ è pari a $X$. In questo caso Y è superchiave per lo schema R(X).

\section{Dipendenze funzionali rilassate}
In alcuni casi per risolvere dei problemi in alcuni di domini di applicazioni, come l’identificazione di inconsistenze tra i dati, o la rilevazione di relazioni semantiche fra i dati,  è necessario rilassare la definizione di dipendenza funzionale, introducendo delle approssimazioni nel confronto dei dati. Invece di effettuare dei controlli di uguaglianza, si utilizzano dei controlli di similarità.
Inoltre spesso si potrebbe desiderare che una certa dipendenza valga solo su un sottoinsieme di tuple che su tutte.
Per questo motivo sono nate delle dipendenze funzionali che rilassano alcuni dei vincoli delle FD, prendono il nome di Dipendenze Funzionali Rilassate o Approssimate \footnote{RFD abbreviazione di Relaxed Functional Dependency.}.
Esistono differenti tipi di RFD, ciascuna di esse rilassa uno o più vincoli delle FD, si possono dividere in due macro aree:
\begin{enumerate}
    \item Confronto di attributi: La funzione di uguaglianza delle FD canoniche viene sostituita da una funzione di similarità , ciò implica che l'AFD deve descrivere una soglia di rilassamento per ogni attributo.
    \item Estensione: Permette che il vincolo non sia valido su tutte le tuple, ma solo su di un sottoinsieme di esse.
\end{enumerate}


Le RFD sono utilizzate in attività di:
data cleaning, record matching e di rilassamento delle query.

Le definizione formale di una RFD è la seguente:
\begin{theorem}
Sia $R$ uno schema relazionale definito su di un insieme di attributi finito, e siano $R_1=(A_1,A_2,...,A_k)$ e $R_2=(B_1,B_2,...,B_m)$ due schemi relazionali definiti su $R$. Una RFD $\varphi$ su $R$ viene rappresentata come:
\\~\\
\centerline{$D_{c_1} \times D_{c_2}:(X_1,X_2)_{\Phi_1} \xrightarrow{\Psi\geq\epsilon}(Y_1,Y_2)_{\Phi_2}$}
\\~\\
dove 
\begin{itemize}
\item $\mathbb{D}_{c_1}\times \mathbb{D}_{c_2} = \{(t_1,t_2)\in dom(R_1)\times dom(R_2)|(\bigwedge_{i=1}^{k} c_{1_i}(t_1[A_i])) \bigwedge_{j=1}^{m} c_{2_j}(t_2[B_j])$, dove $c_{1_i}$ e $c_{2_j}$ sono dei predicati sul $dom(A_i)$ e $dom(B_j)$ rispettivamente, utilizzati per filtrare le tuple a cui $\varphi$ va applicata.
$c_{1_i}(t_1[A_i])$ dà vero se il predicato $c_1$, sull'attributo $A_i$ restituisce vero su $t_1$;
\item $X_1,Y_1 \subseteq attr(R_1)$ e $X_2,Y\subseteq attr(R_2)$ tali che $X_1\cap Y_1=0$ e $X_2\cup Y_2=0 $
\item $\Phi_1$($\Phi_2$ rispettivamente) è un insieme di vincoli $\phi[X_1,X_2](\phi[Y_1,Y_2])$ definite sugli attributi di $X_1$ e $X_2(Y_1$ e $Y_2)$. Per qualsiasi coppia di tuple ($t_1,t_2 \in \mathbb{D}_{c_1} \times \mathbb{D}_{c_2}$) il vincolo $\phi[X_1,X_2](\psi[Y_1,Y_2])$ restituisce vero se la similarità fra $t_1$ e $t_2$ sugli attributi $X_1$ e $X_2(Y_1$ e $Y_2)$ concordano con i vincoli specificati da $\phi[X_1,X_2](\phi[Y_1,Y_2])$;
\item $\Psi: dom(X) \times dom(Y)\xrightarrow{}\mathbb{R}$ rappresenta una misura di copertura su $\mathbb{D}_{c_1} \times \mathbb{D}{c_2}$ e indica il numero di tuple che violano o soddisfano $\varphi$;
\item $\epsilon$ è la soglia che indica il limite superiore o inferiore per il risultato della misura di copertura;
\end{itemize}
\end{theorem}
Nel lavoro di tesi vengono trattate solo le RFD che rilassano il vincolo di uguaglianza. Data RFD $X \xrightarrow{} Y$ essa vale su una relazione $r$ se e solo se la distanza fra due tuple $t_1$ e $t_2$, i cui valori sui singoli attributi $A_i$ non superano una certa soglia $\beta_i$, è inferiore ad una certa soglia $a_A$ su ogni attributo $A \in X$, allora la distanza fra $t_1$ e $t_2$ su ogni attributo $B \in Y$ è minore di una certa soglia $a_B$.\\
La struttura delle RFD utilizzate è la seguente:
\\~\\
\centerline{$attr_1(\leq soglia_1),\ldots$,$attr_n(\leq soglia_n) \xrightarrow{} RHS$}
\\~\\
Gli attributi che si trovano a sinistra della freccia costituiscono la parte LHS\footnote{Left Hand Size o lato sinistro.}, l'attributo che invece si trova dopo la freccia costituisce l'RHS\footnote{Right Hand Size o lato destro.}. 
È importante focalizzare l'attenzione su questo concetto in quanto le dipendenze funzionali hanno un verso, ed è quello indicato dalla freccia. Qualsiasi operazione effettuata con le RFD deve sempre tener conto del verso, le RFD non forniscono conoscenza nel verso opposto. Questa non è una proprietà riguardante solo le RFD, bensì riguarda qualsiasi tipo di dipendenza funzionale.
Ad esempio consideriamo la relazione in questa tabella:
\begin{table}[H]
    \centering
    \begin{tabular}{|l |l |l |l |l |}
    \hline
     Impiegato & Stipendio & Progetto & Bilancio & Funzione \\
    \hline
    Rossi & 20000  & Sito web & 2000 & tecnico\\
    Verdi & 35000 & App Mobile & 15000 & progettista\\
    Verdi & 35000 & Server & 15000 & progettista\\
    Neri & 55000 & Server & 15000 & direttore\\
    Neri & 55000 & App Mobile & 15000 & consulente\\
    Neri & 55000 & Sito web & 2000 & consulente\\
    Mori & 48000 & Sito web & 15000 & direttore\\
    Mori & 48000 & Server & 15000 & progettista\\
    Bianchi & 48000 & Server & 15000 & progettista\\
    Bianchi & 48000 & App Mobile & 15000 & direttore\\
    \hline
    \end{tabular}
    \caption{Esempio di Relazione con anomalie}
    \label{tab:relationship_anomalies}
\end{table}
Si può osservare che lo stipendio di ciascun impiegato è unico, quindi in ogni tupla in cui compare lo stesso impiegato verrà riportato lo stesso stipendio. Possiamo dire che esiste una Dipendenza Funzionale: 
$Impiegato \xrightarrow{} Stipendio$. 
Si può fare lo stesso discorso tra gli attributi Progetto e Bilancio, quindi anche qui abbiamo una dipendenza funzionale
$Progetto\xrightarrow{}Bilancio$. 
Non si può dire che di conseguenza vale anche il verso opposto:
\\~\\
\centerline{$Impiegato \xrightarrow{} Stipendio \neq Stipendio \xrightarrow{} Impiegato$} 
\\~\\
Infatti percepiscono 48000 di stipendio sia Mori che Bianchi.\cite{libroCeri}